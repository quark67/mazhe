% This is part of Exercices et corrigés de CdI-1
% Copyright (c) 2011
%   Laurent Claessens
% See the file fdl-1.3.txt for copying conditions.

\begin{corrige}{0004}

Nous avons
\begin{equation}
	\begin{aligned}[]
		\sup\{ f(x)\tq x\in E \}+\sup\{ g(y)\tq y&\in E \}\\
							&=\sup\{ f(x)+g(y)\tq (x,y)\in E\times E \}\\
									&\geq\sup\{  f(x)+g(y)\tq (x,y)\in E\times E, x=y \} \\
							&=\sup\{ f(x)+g(x)\tq x\in E \}.
	\end{aligned}
\end{equation}
La première inégalité semble un peu mystérieuse. Elle n'est en réalité qu'une application de l'exercice \ref{exo00035} avec
\begin{equation}
	\begin{aligned}[]
		A&=\{ f(x)\tq x\in E \}\\
		B&=\{ g(y)\tq y\in E \}.
	\end{aligned}
\end{equation}

Intuitivement, l'égalité n'aura lieu que quand $f$ et $g$ prennent leur supremum au même point. Prenons par exemple $E=[0,1]$ et puis $f(x)=x$ et $g(x)=1-x$. Le supremum de $f$ et $g$ sont tout deux $1$, tandis que le supremum de la somme est la constante $1$. Donc $\sup\{ f+g \}=1$ tandis que $\sup\{ f \}+\sup\{ g \}=2$.

\vspace{0.5cm}

\noindent{\bf Preuve alternative}\ldots{} due à un étudiant en séance d'exercices.

Prenons $M=\sup\{ f(x)\tq x\in E \}$. Par définition du supremum, $M\geq f(x)$ pour tout $x\in E$. De la même manière, nous prenons $\tilde M=\sup\{ g(y)\tq y\in E \}$. Pour tout $x\in E$, nous avons
\begin{equation}
	f(x)+g(x)\leq M+\tilde M,
\end{equation}
c'est-à-dire que $M+\tilde M$ est un majorant de l'ensemble $\{ f(x)+g(x)\tq x\in E \}$. Étant donné que le supremum est le minimum de l'ensemble des majorants, nous en déduisons que
\begin{equation}
	\sup\{ f(x)+g(x)\tq x\in E \}\leq M+\tilde M.
\end{equation}

\end{corrige}
