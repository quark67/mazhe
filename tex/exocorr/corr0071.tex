% This is part of Exercices et corrigés de CdI-1
% Copyright (c) 2011,2017
%   Laurent Claessens
% See the file fdl-1.3.txt for copying conditions.

\begin{corrige}{0071}

\begin{enumerate}
	\item Une boule ouverte $\{ (x,y) \in \eR^2 \tq x^2+y^2 < 1\}$
	\item Un cercle $\{ (cos(t),sin(t),1) \tq t \in \eR\}$
	\item Pour un exemple non trivial, il faut être dans un espace topologique non connexe. Par exemple dans l'hyperboloïde à deux nappes d'équation $x^2 + y^2 = z^2 - 1$ vu comme espace topologique, la nappe $z > 0$ est ouverte et fermée. L'hyperboloïde lui-même est également un ouvert fermé de lui-même (c'est vrai pour n'importe quel espace topologique).

		Plus généralement, dans un espace topologique non connexe, les composantes connexes sont ouvertes et fermées.
	\item Dans $\eR^n$, n'importe quel ensemble de $8$ points est compact (fermé et borné). Nous pouvons aussi montrer que n'importe quel espace topologique qui possède un nombre fini de points est compact. 

En effet, soit l'ensemble $A=\{ a_1, \ldots, a_n \}$, et prenons un recouvrement de cet ensemble par des ouverts $\mO_{\alpha}$. Comme c'est un recouvrement, nous avons $A\subseteq\bigcup_{\alpha}\mO_{\alpha}$. Soit $\alpha_1$ tel que $a_1\in\mO_{\alpha_1}$ (ceci existe parce que $a_1$ est dans un des $\mO_{\alpha}$). Plus généralement, nous prenons $i$ tel que $a_i$ soit dans $\mO_{\alpha_i}$ pour $i=1, \ldots, n$.

Maintenant, $A\subseteq\bigcup_{i=1}^n\mO_{\alpha_i}$, ce qui fait que nos $\mO_{\alpha_i}$ forment un sous-recouvrement fini de $A$.

	\item Une hélice $\{ (cos(t),sin(t),t) \tq t \in \eR\}$ dans $\eR^3$. Une simple droite dans $\eR^2$ est bon aussi.
	\item Pour obtenir un exemple non trivial (pas $\emptyset$ dans un espace métrique, ni un espace topologique métrique borné vu comme sous-espace de lui-même), il faut se placer dans un espace non connexe. Par exemple dans l'espace topologique formé de l'une union disjointe de deux sphères $S^2 \sqcup S^2$, chaque sphère est ouverte, fermée et bornée.
\end{enumerate}

\end{corrige}
